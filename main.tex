\documentclass{article}
\usepackage[utf8]{inputenc}
\usepackage{amssymb}

\title{\textbf{Lebesgue Measure}}
\author{Abhinav Siddharth \and Akshit Gureja \and Jewel Benny}
\date{January 2021}

\begin{document}

\maketitle

\newpage
\begin{center}
    {\Large\bfseries\noindent Acknowledgement}
\end{center}


add acknowledgement here

\newpage

\begin{center}
    {\Large\bfseries\noindent Abstract}
\end{center}

add abstract here
\newpage

\begin{center}
    \tableofcontents
\end{center}

\newpage

\section{Introduction}
The goal of this project is to investigate the Lebesgue measure and some of its applications in probability measure. 

\subsection{Measure}

How do we measure a set or an object? If we have a 3-dimensional object, we could find its volume in 2 different ways, we could fill the container with simple objects or $boxes$ and find their sum, or we could encase the whole container in a box and start chipping away smaller boxes. These approaches work only when our box measure is a properly defined measure. 

The mathematical concept of measure is a generalisation of length in $\mathbb{R}$, area in $\mathbb{R}^2$ or volume in $\mathbb{R}^3$. To avoid splitting into cases depending on the dimension, we shall refer to the $measure$ of E, depending on what Euclidean space $\mathbb{R}^n$ we are using and $E \subset \mathbb{R}^n$.

We know the measure or length of an interval, say $[0,1]$, we also have $(0,1)$ and $(0,1]$, as these are subsets of our initial interval, their measures must not exceed that of $[0,1]$ (monotonic). Let us assume the length of this unit interval be 1. We let the measure to be zero, for a single point or an empty set, and infinite if we consider the real line. Ideally, we'd like to associate a non-negative measure $m(E)$ to every subset E of $\mathbb{R}^n$. It should also obey some reasonable properties like $m(A \cup B) = m(A) + m(B)$ whenever A and B are disjoint, we should have $m(A) \leq m(B)$ whenever $A \subset B$, and $m(x+A)=m(A)$ (translation invariant). Here, we hit a roadblock in our definition of a measure:
$$(0,1)= \bigcup_{x \in (0,1)} \{x\} \, and \, 1 \neq \sum_{(0,1)}0$$

From our definition, we can also see that two sets having the same number of points need not have the same measure, let $A=[0,1]$ and $B=[0,2]$, there exists a bijection from A to B (x $\mapsto$ 2x), but B is twice as long as A.
 
 Remarkably, it turns out such a measure $does \, not \,exist$. This is quite a surprising fact, because it is counter-intuitive. These examples tell us that it is impossible to measure every subset of $\mathbb{R}^n$ applying all the above properties. 
 
 If we are developing a measure $m$ defined on the subsets of $\mathbb{R}$, we hope that these conditions are met:
 \begin{enumerate}
     \item $m(A)$ is defined for every set A of the real numbers;
     \item $0\leq m(A) \leq \infty$;
     \item $m(A) \leq m(B)$ provided $A \subset B$;
     \item $m(\phi) =0 $;
     \item $m(\{a\})=0$ (points are dimensionless);
     \item $m(I)=l(I)$, I is an interval (the measure on an interval should be its length)
     \item $m(A)=m(x+A)$, translation invariance, (location doesn't affect the length, so it shouldn't affect measure)
     \item $m(\bigcup_{1}^{\infty} {A_k}) = \sum_{1}^{\infty}m(A_k)$, for any mutually disjoint sequence $(A_k)$ of subsets of real numbers (countable additivity).
 \end{enumerate}
 Our goal is to construct a measure that satisfies as many conditions as possible.
 
 \subsubsection{Length of Intervals}
 
 The length of an interval I with end points $a\leq b$, $a,b \in \mathbb{R}$, is defined as:
 $$l(I) = b-a$$
 For example, $l((0,1])=l([0,1])=1$ and $l((1,\infty))=\infty$.
 
 We immediately conclude that, if $I_1$ and $I_2$ are two intervals with $I_1 \subset I_2$, $l(I1) \leq l(I2)$.
 
 \subsubsection{Cover}
 
 A collection $\{G_{\alpha}\}$ of open sets covers a set A if $A \subset \bigcup G_{\alpha}$. And the collection $\{G_{\alpha}\}$ is called the cover.
 
 \subsubsection{Subcover}
 
 Let C be a cover of a topological space X. A subcover of C is a subset of C that still covers X.
 
 \subsubsection{Compact Set}
 
A set of real numbers is compact if every open cover of the set contains a finite subcover.

\subsubsection{Heine-Borel Theorem}
A set of real numbers is compact iff it is closed and bounded.

\subsubsection{Result}
If $I,I_1,I_2,\dots ,I_n$ are bounded open intervals with 

$$ I \subset \bigcup_{1}^{n}I_k ,\; then  \;l(I) \leq \sum l(I_k)$$

The length of an interval can not exceed the length of a $finite$ cover.

\subsubsection{Proposition}

If $I,I_1,I_2,\dots ,I_n$ are bounded open intervals with $ I \subset \bigcup I_k$, then $l(I) \leq \sum l(I_k)$. Also, $ l(I) \leq inf\{ \sum l(I_k) | I \subset \bigcup I_k, I_k \; bounded \; intervals \}$.

Proof: Assume $I =(a,b)$ and let $\epsilon > 0$. The intervals $(a-\epsilon,a+\epsilon),(b+\epsilon,b-\epsilon),I_1,\dots ,I_n$ form an open cover of the compact set $[a,b]$. And by Heine-Borel Theorem, a finite sub collection will cover [a,b] and thus (a,b).
Using 1.1.6:
$$  l(I) \leq 4\epsilon +  \sum l(I_k)$$

As this holds for any $\epsilon$, $l(I) \leq \sum l(I_k)$, the proof is complete.

\subsection{Lebesgue Outer Measure}

We have seen that every set of real numbers can be covered with a countable collection of open intervals.

\subsubsection{Definition}

A is any subset of $\mathbb{R}$. Form the collection of all countable covers of A by open intervals. The Lebesgue outer measure of A, $m^*(A)$, is given by
$$m^*(A)= inf\Big\{\sum_{1}^{\infty}l(I_k)| A \subset \bigcup_{1}^{\infty} I_k \, , \; I_k \; open \; intervals\Big\}$$

\subsubsection{Results}
\begin{enumerate}
    \item $m^*$ is a set function, whose domain is all subsets of $\mathbb{R}$, and range is $[0,\infty]$, the non-negative extended real numbers. Note that the Lebesgue Outer Measure is defined for every subset of $\mathbb{R}$.
    
    \item Outer Measure is monotonic, i.e. , if $A \subseteq B$, then $m^*(A) \leq m^*(B)$, any cover of B by open intervals is also a cover of A, and the latter infimum is taken over a larger collection than the former. 

    \item If $(I_k)$ is any countable cover of A by open intervals, since infimum is a lower bound,$$ 0 \leq m^*(A) \leq \sum l(I_k)$$
    


\end{enumerate}

\subsubsection{The outer measure of any interval is it's length.}

Proof:  
For a closed interval [a,b], let $\epsilon >0$. Then $(a-\epsilon,b+\epsilon)$ covers $[a,b]$ and length of this interval is $b-a+\epsilon$. Since $\epsilon$ is arbitrary $m^*([a,b]) \leq b-a = l([a,b])$.

Next let ${I_n}$ be a covering of [a,b] by bounded open intervals. By the Heine-Borel Theorem, there exists a finite subset A of $I_n$'s covering [a,b]. So $a \in I_1$ for some $I_1 =(a_1,b_1) \in A$. Also, if $b_1 \leq b$, then $b_1 \in I_2 $ for some $I_2 =(a_2,b_2) \in A$. Similarly we can construct $I_1,I_2,\dots,I_k$. Then
$$ \sum l(I_n) \geq \sum_{i=1}^k l(I_i) = \sum_{i=1}^{k}(b_i - a_i)$$
$$=(b_k-a_k) +(b_{k-1} -a_{k-1})+ \dots + (b_1 - a_1) $$
$$ >b_k -a_1$$
Since $a_1<a$ and $b_k>b$, then $ \sum l(I_n) > b_k -a_1 $. So, $m^*([a,b]) = b-a = l([a,b])$.

If I is an unbounded interval, then given any natural number $n \in N$, there is a closed interval $J \subset I$ with $l(J)=n$. Hence $m^*(I) \geq m^*(J) = l(J)=n$. Since $m^*(I) \geq n \, and \, n \in N$ is arbitrary, then $m^*(I)=\infty=l(I)$. 

\subsubsection{Outer Measure is translation invariant}
$m^*(A)=m^*(A+y)$

Proof: Suppose $m^*(A) = M < \infty$. Then for all $\epsilon > 0 $ there exist ${I_n}$ bounded open intervals, such that they cover A. And from sec 1.1.7. $\sum l(I_n) < M+ \epsilon$, so if $y \in \mathbb{R}$, then ${I_{n}+y}$ is a covering of A+y and so $m^{*}(A+y) \leq \sum l(I_n + y ) = \sum l(I_n) < M + \epsilon$ .Therefore $m^*(A+y) \leq M$.

Now, let ${J_n}$ be a collection of bounded open intervals such that $\cup J_n \supset A+y $.Assume that $\sum l(J_n)< M$. Then ${J_n-y}$ is a covering of A and $\sum l(J_n-y)=\sum l(J_n) < M$, a contradiction. So, $\sum l(J_n) \geq M$ and hence $m^*(A)\geq M$. So $m^*(A)=m^*(A+y)=M$

Suppose $m^*(A)= \infty$.Then for any$\{I_n\}_{n=1}^{\infty} $ a set of bounded open intervals such that $A \subset \cup I_n$, we must
have$\sum l(I_n) =\infty$ .Consider A+y. For any $\{J_n\}_{n=1}^{\infty} $ a set of bounded open intervals such that $A+y \subset \cup J_n$, the collection $\{J_n - y\}_{n=1}^{\infty} $ is a set of bounded open intervals such that $A+y \subset \cup J_n -y$. So $\sum l(J_n - y) =\infty$. But l($J_n$) =l($J_n - y$), so we must have $\sum l(J_n) =\infty$. Since $\{J_n\}_{n=1}^{\infty} $ is an arbitrary collection of bounded open intervals covering A+y, we must have $m^*(A)=m^*(A+y)= \infty$. 

\subsubsection{Outer Measure is countably sub-additive}

That is, if $\{E_k\}_{k=1}^{\infty} $ is any countable collection of sets, then
$$ m^*(\bigcup_{k=1}^{\infty} E_k) \leq \sum_{k=1}^{\infty} m^{*}(E_k)$$

Proof: If one of the $E_k$'s has infinite outer measure, the inequality holds trivially. We therefore suppose each of the $E_k$'s has finite outer measure. Let $\epsilon > 0$. For each natural number k, there is a countable collection $\{ I_{k,i} \}_{i=1}^{\infty}$ of open bounded intervals for which
$$ E_k \subseteq \bigcup_{i=1}^{\infty} \{ I_{k,i} \} \; and \; \sum_{i=1}^{\infty} l(I_{k,i}) < m^{*}(E_k) +\frac{\epsilon}{2^k}  $$

Now, $\{I_{k,i}\}_{1 \leq k , i \leq \infty}$ is a countable collection of open, bounded intervals that covers $\cup_{k=1}^{\infty} E_k$: the collection is countable since it is a countable collection of countable collections. Thus,

$$ m^{*}(\bigcup_{k=1}^{\infty} E_k) \leq \sum_{1\leq k,i<\infty } l(I_{k,i})= \sum_{k=1}^{\infty}[\sum_{i=1}^{\infty} l(I_{k,i})]$$
$$\;\;\;<\sum_{k=1}^{\infty}[m^{*}(E_k) + \frac{\epsilon}{2^k}]$$
$$  m^{*}(\bigcup_{k=1}^{\infty} E_k) \leq [\sum_{k=1}^{\infty}m^{*}(E_k)]+\epsilon$$

Since this holds for each $\epsilon>0$, it also holds for $\epsilon=0$. The proof is complete.

The finite sub-additivity property follows from countable sub-additivity by taking $E_k = \phi$ for $k>n$.

\subsubsection{$[0,1]$ is not countable}

As seen in Sec.1.2.3 $[0,1]$ has an outer measure of 1, whereas all countable sets have a measure 0, (refer to Royden and Fitzpatrick pg.31), therefore $[0,1]$ is not countable.


\subsection{$\sigma$-Algebra of Lebesgue Measurable sets}

Unfortunately the outer measure fails to be countably additive on all subsets of $\mathbb{R}$, it is not even finitely additive. A famous example is the Vitali Set, found by Giuseppe Vitali in 1905. However, we can salvage matters by measuring a certain class of sets in $\mathbb{R}$, henceforth called the measurable sets. Once we restrict ourselves to measurable sets, we recover all the properties again.

\subsubsection{Carathedory's Measurability Criteria}

E is any set of real numbers. If

$$m^*(X)= m^*(X \cap E) + m^*(x \cap E^c) $$
for every set X of real numbers, then the set E is said to be Lebesgue Measurable set of real numbers.

\subsubsection{Results}
\begin{enumerate}
    \item Any set of outer measure is measurable, in other words, any countable set is measurable.
    \item The union of a finite collection of measurable sets is measurable. This can be inferred using set identities and finite sub-additivity property of outer measure, then applying induction.
    \item The union of countable collection of measurable sets is measurable.
\end{enumerate}

\subsubsection{Definition of $\sigma$-Algebra}
A $\sigma$-algebra (or $\sigma$-field) $\Sigma$ on a set $\Omega$ is a subset of $P(\Omega)$ (the power set of $\Omega$) such that it satisfies the following properties -


\begin{enumerate}
	\item $\Omega$ $\in$ $\Sigma$ \textit{(property 1)}
	\item If A is an element of $\Sigma$ then its complement also is an element of $\Sigma$ i.e A $\in$ $\Sigma$ $\Rightarrow$ A$^c$ $\in$ $\Sigma$ \textit{(\textit{property 2})}
	\item If $A_i \in \Sigma$, 
	$i\in \mathbb{N}$, then $\bigcup\limits_{i=1}^{n}A_i \in \Sigma$\textit{(\textit{property 3})}
\end{enumerate}



\subsubsection{Examples}
\begin{enumerate}
	\item Consider $\Sigma$ $=$ \{$\phi$, $\Omega$\}. The given set is a $\sigma$-algebra on $\Omega$ as it satisfies all the properties needed. Observe that $\phi$ is also included in $\Sigma$ by \textit{property 2}. \textit{Property 1} is also satisfied by the given $\sigma$-algebra. $\phi $ $\cup$ $\Omega$ = $\Omega$ $\in \Sigma$, so \textit{property 3} is also satisfied.
	\item Consider $\Sigma$ = $P(\Omega)$. Then, $\Sigma$ satisfies \textit{property 1} as $\Omega$ $\in$ $\Sigma$. Furthermore, $X^c$ = $\phi$ $\in$ $\Sigma$ and hence it satisfies \textit{property 2}. As $\Sigma$ contains all the subsets of $\Omega$, it is guaranteed that the union of any number of subsets of $\Omega$ exists in $\Sigma$, and hence satisfying \textit{property 3}. Hence $\Sigma$ = $P(\Omega)$ is a $\sigma$-algebra. 
	\item Consider $\Sigma$ = \{$\phi$, \{1\}, \{2\}, \{1, 2\}, $\Omega$\} and $\Omega$ = \{1, 2, 3\}. Here, $\Sigma$ is \textbf{not} a $\sigma$-algebra on $\Omega$ as \{1, 2\}$^c$ = \{3\} is not an element of $\Sigma$, and hence $\Sigma$ does not satisfy \textit{property 2}.
\end{enumerate}


\subsubsection{Proposition 1}
If $\Sigma$ is a $\sigma$-algebra on $\Omega$, and A$_1$, A$_2$, A$_3$,....A$_n$ $\in$ $\Sigma$, then $\bigcap\limits_{i=1}^n$A$_i$ $\in$ $\Sigma$.

\textbf{Proof: }If A$_1$, A$_2$, A$_3$,....A$_n$ $\in$ $\Sigma$, then A$_1^c$, A$_2^c$, A$_3^c$,....A$_n^c$ $\in$ $\Sigma$ and $\bigcup\limits_{i=1}^n$A$_i^c$ $\in$ $\Sigma$. Then by \textit{property 2}, $(\bigcup\limits_{i=1}^n$A$_i^c)^c$ = $\bigcap\limits_{i=1}^n$A$_i$ $\in$ $\Sigma$.


\subsubsection{Proposition 2}
If $\Sigma$ is a $\sigma$-algebra on $\Omega$ and A, B $\in$ $\Sigma$, then A$\setminus$B = A - B $\in$ $\Sigma$.

\textbf{Proof: }If B $\in$ $\Sigma$, then B$^c$ $\in$ $\Sigma$ by  \textit{property 2}. So, A $\cap$ B$^c$ = A$\setminus$B $\in$ $\Sigma$ (As seen in proposition 1).


\subsubsection{Proposition 3}
If $\Sigma_i$ is a $\sigma$-algebra on $\Omega$, i $\in \mathbb{N}$, then $\bigcap\limits_{i=1}^n$ $\Sigma_i$ is also a $\sigma$-algebra on $\Omega$.

\textbf{Proof: }Consider the $\sigma$-algebras $\Sigma_1$, $\Sigma_2$, $\Sigma_3$,... $\Sigma_n$ defined on $\Omega$, then the smallest intersection of all such $\sigma$-algebras would be \{$\phi$, $\Omega$\}, which is a $\sigma$-algebra by definition. Also, if A is an element of the intersection, then A$^c$ would also be an element of the intersection, which satisfies \textit{property 2}. Also, A $\cup$ A$^c$ = $\Omega$ $\in$ $\Sigma$ which satisfies properties 1 and 3.



\subsubsection{Proposition 4}
For $Y$ $\subseteq$ P($\Omega$), there exists a smallest $\sigma$-algebra that contains $Y$, called the $\sigma$-algebra generated by $Y$, given by the intersection of all such $\sigma$-algebras $\Sigma$ such that $\Sigma \supseteq Y$. That is, $\sigma$($Y$) = $\bigcap\limits_{\Sigma \supseteq Y}\Sigma$.

\textbf{Proof: }Consider $\sigma$-algebras $\Sigma_1$, $\Sigma_2$, $\Sigma_3$,....  $\Sigma_n$ $\supseteq $ $Y$ defined on $\Omega$. Then $\bigcap\limits_{i=1}^n \Sigma_i \supseteq Y$. As $Y \in$ $\Sigma_i$, $Y^c \in \Sigma_i$ for i = 1, 2, 3...n which satisfies \textit{property 2}. Furthermore, $\Omega$ $\in \Sigma_i$ which satisfies \textit{property 1}. $Y \cup Y^c$ = $\Omega$ which satisfies \textit{property 3}. Hence $\sigma(Y)$ is a $\sigma$-algebra.

For example, consider $\Omega$ = \{1, 2, 3, 4\} and $Y$ = \{\{1\}, \{2\}\}. Then the smallest $\sigma$-algebra on $\Omega$ containing $Y$ is $\sigma(Y)$ = \{$\phi$, \{1\}, \{2\}, \{1, 2\}, \{3, 4\}, \{1, 3, 4\}, \{2, 3, 4\}, $\Omega$\}. Also note that if $Y$ is a $\sigma$-algebra, then $\sigma(Y)$ = $Y$.

\subsubsection{Every interval is measurable}
From the preceding propositions, we immediately conclude that the collection of measurable sets is a $\sigma$-algebra. Also, Lebesgue outer measure $m^*$, written as $m$ when restricted to the $\sigma$-algebra $E$ subsets of $R$ satisfying Carathedory's condition, is countably additive on $E$, in other words, $m$ is countably additive on Lebesgue measurable subsets of R.

It can further be shown Intervals satisfy Caratheodory's condition, using the Heine-Borel theorem. But we can use Borel sets to extend our applicability of Lebesgue measure to a much larger group of subsets.

\subsection{Borel Sets}
The $\sigma$-algebra generated by the collection of all open intervals of R is called the Borel $\sigma$-algebra $\mathbb{B}$.

Therefore, since the measurable sets are a $\sigma$-algebra containing all open sets, every Borel set of real numbers is Lebesgue measurable. $\mathbb{B}$ contains about every set that comes up in analysis.

\subsubsection{Contents of Borel Sets}
\begin{itemize}
    \item Open intervals
    \item Open sets
    \item Closed intervals
    \item Closed sets
    \item Compact sets
    \item Left open, right closed sets
    \item Right open, left closed sets
    \item All intervals
\end{itemize}

Every Borel set is Lebesgue measurable, but not vice-versa, there exist sets that are Lebesgue measurable which are not Borel sets.
%add examples here
The collection of all Lebesgue measurable sets is a $\sigma$-algebra $\mathbb{M}$, and $\mathbb{B}$, the collection of Borel sets, then
$$\mathbb{B} \subset \mathbb{M} \subset 2^{\mathbb{R}} $$


\subsection{Lebesgue Measure}

By restricting outer measure of to a class of measurable sets, we can define the Lebesgue measure. It is denoted by $m$, if E is a measurable set, it's Lebesgue measure is defined to be
$$m(E)=m^*(E)$$

We can use all our previous results to determine the Lebesgue measure of specific sets of real numbers, note that all following sets are presumed to be Lebesgue measurable sets of real numbers
\subsubsection{Theorem}
\begin{enumerate}
    \item $m(\phi)$=$m(\{a\})$ = 0.
    \item $m(I)$ = $l(I)$.
    \item $m$(countable set) = 0.
    \item $m$(subset of a set with measure 0) = 0.
    \item $m(\cup E_k) \leq \sum m(E_k)$ (equal when $E_k$ are disjoint).
    \item $m(E_1 \cup E_2) + m(E_1 \cap E_2) = m(E_1)+ m(E_2)$.
    \item $m(E_1) \leq m(E_2)\, if \, E_1 \subset E_2$. If in addition, $m(E_2) <\infty$, then $m(E_2)-m(E_1)=m(E_2-E_1)$.
    \item If $E_1 \subset E_2 \subset E_3 \dots$ then $m(\cup E_k)=m(\lim E_k)=\lim m(E_k)$
    \item If $E_1 \supset E_2 \supset E_3 \dots$ and $m(E_1) < \infty $then $m(\cap     E_k)=m(\lim E_k)=\lim m(E_k)$
    \item If $m(\cup E_k) < \infty$, then $\lim sup m(E_k) \leq m(\lim sup E_k)$
    \item  $ m(\lim inf E_k) \leq \lim inf m(E_k) $
    \item If $\lim inf E_k = \lim sup E_k$ and $m(\cup E_k)<\infty$, then $m(\lim E_k)=\lim m(E_k)$
\end{enumerate}
%add examples here too, find lebesgue measure of some sets





\end{document}
